\documentclass[10pt,letterpaper]{article}
\usepackage[utf8]{inputenc}
\usepackage[english]{babel}
\usepackage{amsmath}
\usepackage{amsfonts}
\usepackage{amssymb}
\usepackage{graphicx}
\usepackage{courier}
\usepackage{booktabs}
\usepackage{tabularx}
\usepackage{microtype}
% \usepackage[T1]{fontenc}
% \renewcommand\tabularxcolumn[1]{m{#1}}% vertically center
\newcolumntype{Y}{>{\raggedright\arraybackslash}X} % A regular tabularx column but left justified instead of full justified.
% -------------------------------------------------------
% FROM https://tex.stackexchange.com/questions/348651/c-code-to-add-in-the-document
\usepackage{listings}
\usepackage{xcolor}
\definecolor{mGreen}{rgb}{0,0.6,0}
\definecolor{mGray}{rgb}{0.5,0.5,0.5}
\definecolor{mPurple}{rgb}{0.58,0,0.82}
\definecolor{backgroundColour}{rgb}{0.95,0.95,0.92}

\lstdefinestyle{CStyle}{
  commentstyle=\color{mGreen},
  keywordstyle=\color{magenta},
  numberstyle=\tiny\color{mGray},
  stringstyle=\color{mPurple},
  basicstyle=\footnotesize\ttfamily,
  breakatwhitespace=false,         
  breaklines=true,                 
  captionpos=b,       
  keepspaces=true,
  numbersep=5pt,
  numbers=left,
  showspaces=false,                
  showstringspaces=false,
  showtabs=false,                  
  tabsize=2,
  language=C
}

\lstset{style=CStyle}
% -------------------------------------------------------
% \usepackage[left=2cm,right=2cm,top=2cm,bottom=2cm]{geometry}


% BUG REPORT TEMPLATE -----------------------------------
%% \noindent\begin{tabularx}{\textwidth}{rY}
%%   \toprule
%%   File & \\
%%   Line & \\
%%   Description & \\
%%   Number of Configurations & \\
%%   \midrule
%%   \multicolumn{2}{c}{Code Sample} \\
%% \end{tabularx}
%% \noindent\lstinputlisting[linerange=, firstnumber=]{code/}
%% \noindent\begin{tabularx}{\textwidth}{rY}
%%   \midrule
%%   Status & \\
%%   Remarks & \\
%%   \bottomrule
%% \end{tabularx}
% -------------------------------------------------------

\title{Toybox Bug Analysis}
\author{Austin Mordahl}
\begin{document}
\maketitle
\tableofcontents

\section{Introduction}
\noindent These bugs were generated by Cppcheck 1.72 and Toybox 0.7.5.
Bug reports are classified into the following categories:

\begin{center}
  \noindent\begin{tabularx}{0.75\textwidth}{rX}
    \toprule
    True & A bug which exists and 1) its existence is unintended, or 2) whether or not its existence is purposeful is undetermined. \\
    \midrule
    Technically True & A bug for which the content of the cppcheck bug report is true, but whose existence is intended. The difference between a False and Technically True bug report is that the former could theoretically be detected by a more sophisticated implementation of cppcheck. \\
    \midrule
    False & A bug cppcheck finds which, upon further inspection, does not exist in the code. For example, cppcheck indicating a variable is passed to a function without being initialized, when the variable is actually an out parameter and intialized within the function.\\
    \bottomrule
  \end{tabularx}
\end{center}


\pagebreak
\section{True Reports}
\noindent\begin{tabularx}{\textwidth}{rY}
  \toprule
  File & blockdev.c \\
  Line & 60 \\
  Description & Array \texttt{cmds[11]} accessed at index 31, which is out of bounds.\\
  Number of Configurations & 482 \\
  \midrule
  \multicolumn{2}{c}{Code Sample} \\
\end{tabularx}
\noindent\lstinputlisting[linerange=52-64, firstnumber=52]{code/blockdev.c}
\noindent\begin{tabularx}{\textwidth}{rY}
  \midrule
  Status & Likely False \\
  Remarks & \texttt{cmd[]} is defined as an integer array of size
  11. By using a loop that iterates through the number 31 to access
  the loop, the program is exceeding the bounds of the array.
  However, there may be checks on toys.optflags elsewhere in the program
  that prevent i from ever exceeding the array bounds.

  It appears the comments above blockdev code (and at the beginning of
  every toy) defines the flags and options, so as long as that's right
  the loop won't exceed the bounds of the command array.
  \\
  Features & CONFIG\_BLOCKDEV \\
  \bottomrule
\end{tabularx}

\pagebreak

\noindent\begin{tabularx}{\textwidth}{rY}
  \toprule
  File & netstat.c\\
  Line & 118\\
  Description & {Resource leak: \texttt{fp}}\\
  Number of Configurations & {515}\\
  \midrule
  \multicolumn{2}{c}{Code Sample} \\
\end{tabularx}
\noindent\lstinputlisting[linerange=111-118, firstnumber=111]{code/netstat.c}
\noindent\begin{tabularx}{\textwidth}{rY}
  \midrule
  Status & True\\
  Remarks & \texttt{fp} is not closed before the function returns.\\
  Features & CONFIG\_NETSTAT \\
  \bottomrule
\end{tabularx}

\pagebreak	

\noindent\begin{tabularx}{\textwidth}{rY}
  \toprule
  File & cmp.c \\
  Line & 83 \\
  Description & Signed integer overflow for expression \texttt{(2147483648)*!(toys.optflags\&(1))}. \\
  Number of Configurations & 501 \\
  \midrule
  \multicolumn{2}{c}{Code Sample} \\
\end{tabularx}
\noindent\lstinputlisting[linerange=80-85, firstnumber=80]{code/cmp.c}
\noindent\begin{tabularx}{\textwidth}{rY}
  \midrule 
  Status & False \\
  Remarks & \texttt{!} is logical negation and not bitwise negation,
  so WARN_ONLY will only be multiplied by 1 or 0. \\
  Features & CONFIG\_CMP \\
  \bottomrule
\end{tabularx}

\pagebreak

\section{Technically True Reports}
\noindent\begin{tabularx}{\textwidth}{rY}
  \toprule
  File & chvt.c \\
  Line & 24 \\
  Description & Uninitialized variable: \texttt{fd} \\
  Number of Configurations & 512 \\
  \midrule
  \multicolumn{2}{c}{Code Sample} \\
\end{tabularx}
\noindent\lstinputlisting[linerange=22-34, firstnumber=22]{code/chvt.c}
\noindent\begin{tabularx}{\textwidth}{rY}
  \midrule
  Status & Technically True \\
  Remarks & The self-assignment \texttt{fd=fd} is likely purposeful, as a method to suppress compiler warnings about an unused variable \texttt{fd} before the rest of \texttt{chvt\_main} was written to use \texttt{fd}. However, cppcheck is correct in that \texttt{fd\= fd} is an assignment of the value of an uninitialized variable.\\
  Feature & CONFIG\_CHVT \\
  \bottomrule
\end{tabularx}

\pagebreak

\noindent\begin{tabularx}{\textwidth}{rY}
  \toprule
  File & date.c \\
  Line & 137 \\
  Description & Uninitialized variable: \texttt{width}\\
  Number of Configurations & 511 \\
  \midrule
  \multicolumn{2}{c}{Code Sample} \\
\end{tabularx}
\noindent\lstinputlisting[linerange=134-160, firstnumber=134]{code/date.c}
\noindent\begin{tabularx}{\textwidth}{rY}
  \midrule
  Status & Technically True \\
  Remarks & See the report for \texttt{chvt.c:24}.\\
  Features & CONFIG\_DATE \\
  \bottomrule
\end{tabularx}

\pagebreak
\noindent\begin{tabularx}{\textwidth}{rY}
  \toprule
  File & hwclock.c\\
  Line & 89\\
  Description & Uninitialized variable: \texttt{s}\\
  Number of Configurations & 466\\
  \midrule
  \multicolumn{2}{c}{Code Sample} \\
\end{tabularx}
\noindent\lstinputlisting[linerange=88-98, firstnumber=88]{code/hwclock.c}
\noindent\begin{tabularx}{\textwidth}{rY}
  \midrule
  Status & Technically True\\
  Remarks & See the report for \texttt{chvt.c:24}.\\
  Features & CONFIG\_HWCLOCK \\
  \bottomrule
\end{tabularx}

\pagebreak

\noindent\begin{tabularx}{\textwidth}{rY}
  \toprule
  File & losetup.c\\
  Line & 64\\
  Description & Uninitialized variable: \texttt{ffd}\\
  Number of Configurations & 531\\
  \midrule
  \multicolumn{2}{c}{Code Sample} \\
\end{tabularx}
\noindent\lstinputlisting[linerange=63-69, firstnumber=63]{code/losetup.c}
\noindent\begin{tabularx}{\textwidth}{rY}
  \midrule
  Status & Technically True\\
  Remarks & See the report for \texttt{chvt.c:24}.\\
  Features & CONFIG\_LOSETUP \\
  \bottomrule
\end{tabularx}

\pagebreak

\noindent\begin{tabularx}{\textwidth}{rY}
  \toprule
  File & switch\_root.c\\
  Line & 49\\
  Description & Uninitialized variable: \texttt{console}\\
  Number of Configurations & 486\\
  \midrule
  \multicolumn{2}{c}{Code Sample} \\
\end{tabularx}
\noindent\lstinputlisting[linerange=46-49, firstnumber=46]{code/switch_root.c}
\noindent\lstinputlisting[linerange=81-84, firstnumber=81]{code/switch_root.c}
\noindent\begin{tabularx}{\textwidth}{rY}
  \midrule
  Status & Technically True\\
  Remarks & See the report for \texttt{chvt.c:24}.\\
  Features & CONFIG\_SWITCH\_ROOT \\
  \bottomrule
\end{tabularx}

\pagebreak

\noindent\begin{tabularx}{\textwidth}{rY}
  \toprule
  File & uudecode.c\\
  Line & 29\\
  Description & Uninitialized variable: \texttt{m}\\
  Number of Configurations & 485\\
  \midrule
  \multicolumn{2}{c}{Code Sample} \\
\end{tabularx}
\noindent\lstinputlisting[linerange=29-43, firstnumber=29]{code/uudecode.c}
\noindent\begin{tabularx}{\textwidth}{rY}
  \midrule
  Status & Technically True\\
  Remarks & See the report for \texttt{chvt.c:24}.\\
  Features & CONFIG\_UUDECODE \\
  \bottomrule
\end{tabularx}

\pagebreak

\noindent\begin{tabularx}{\textwidth}{rY}
  \toprule
  File & vmstat.c\\
  Line & 51\\
  Description & Uninitialized variable: \texttt{name}\\
  & Uninitialized variable: \texttt{p}\\
  Number of Configurations & 508\\
  \midrule
  \multicolumn{2}{c}{Code Sample} \\
\end{tabularx}
\noindent\lstinputlisting[linerange=51-70, firstnumber=51]{code/vmstat.c}
\noindent\begin{tabularx}{\textwidth}{rY}
  \midrule
  Status & Technically True\\
  Remarks & See the report for \texttt{chvt.c:24}.\\
  Features & CONFIG\_VMSTAT \\
  \bottomrule
\end{tabularx}

\pagebreak

\section{False Reports}

\noindent\begin{tabularx}{\textwidth}{rY}
  \toprule
  File & lsm.h \\
  Line & 63 \\
  Description & Uninitialized variable: result \\
  Number of Configurations & 432\footnote{The actual cppcheck bug reports listed various C source code files which included this header as the source of the bug, even though \texttt{lsm.h} was the actual source. This is the number of total occurrences of the bug across multiple files.}\\
  \midrule
  \multicolumn{2}{c}{Code Sample} \\
\end{tabularx}
\noindent\lstinputlisting[linerange=55-64, firstnumber=55]{code/lsm.h}
\noindent\begin{tabularx}{\textwidth}{rY}
  \midrule
  Status & False\\
  Remarks & In configurations including \texttt{TOYBOX\_SMACK} and \texttt{TOYBOX\_SELINUX} \texttt{smack\_new\_label\_from\_self} and \texttt{getcon} are replaced with the value -1, respectively. In other configurations, \texttt{*result} is an out parameter. \\
  \bottomrule
\end{tabularx}

\pagebreak

\noindent\begin{tabularx}{\textwidth}{rY}
  \toprule
  File & base64.c\\
  Line & 35\\
  Description & Expression \texttt{`this.base64.columns\&\&++*x == this.base64.columns'} \\ & depends on order of evaluation of side effects. \\
  Number of Configurations & 478 \\
  \midrule
  \multicolumn{2}{c}{Code Sample} \\
\end{tabularx}
\noindent\lstinputlisting[linerange=31-39, firstnumber=31]{code/base64.c}
\noindent\begin{tabularx}{\textwidth}{rY}
  \midrule
  Status & False\\
  Remarks & Although \texttt{TT.columns} appears twice in the same expression, it is modified neither time. Thus, the order of evaluation of side effects does not matter. \\
  \bottomrule
\end{tabularx}

\pagebreak

\noindent\begin{tabularx}{\textwidth}{rY}
  \toprule
  File & tail.c\\
  Line & 188\\
  Description & Memory is allocated but not initialized: \texttt{try}\\
  Number of Configurations & 655\\
  \midrule
  \multicolumn{2}{c}{Code Sample} \\
\end{tabularx}
\lstinputlisting[firstline=181,lastline=188,firstnumber=181]{code/tail.c}
\noindent\begin{tabularx}{\textwidth}{rY}
  \midrule
  Status & False\\
  Remarks & The \texttt{for} loop causing cppcheck to give a warning is actually only testing \texttt{try[count]} for equality.\\
  \bottomrule
\end{tabularx}

\pagebreak

\noindent\begin{tabularx}{\textwidth}{rY}
  \toprule
  File & args.c\\
  Line & 309\\
  Description & Uninitialized variable: \texttt{temp}\\
  Number of Configurations & 519\\
  \midrule
  \multicolumn{2}{c}{Code Sample} \\
\end{tabularx}
\noindent\lstinputlisting[firstline=301, lastline=309, firstnumber=301]{code/args.c}
\noindent\begin{tabularx}{\textwidth}{rY}
  \midrule
    Status & False\\
   Remarks & \texttt{temp} is necessarily initialized by either a call to \texttt{strol} or \texttt{strod}.\\
  \bottomrule
\end{tabularx}

\pagebreak

\begin{tabularx}{\textwidth}{rY}
  \toprule
  File & lib.c\\
  Line & 975\\
  Description & Buffer is accessed out of bounds: \texttt{``xwr''}\\
  Number of Configurations & 986\\
  \midrule
  \multicolumn{2}{c}{Code Sample} \\
\end{tabularx}
\noindent\lstinputlisting[linerange=965-977, firstnumber=965]{code/lib.c}
\noindent\begin{tabularx}{\textwidth}{rY}
  \midrule
  Status & False\\
  Remarks & \texttt{``xwr''} will never be accessed out of bounds. \texttt{c} is assigned by the expression \texttt{c = i \% 3}, which will give \texttt{c} the value of 0, 1, or 2.\\
  \bottomrule
\end{tabularx}

\pagebreak

\end{document}
