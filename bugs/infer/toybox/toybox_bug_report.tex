\documentclass[10pt,letterpaper]{article}
\usepackage[utf8]{inputenc}
\usepackage[english]{babel}
\usepackage{amsmath}
\usepackage{amsfonts}
\usepackage{amssymb}
\usepackage{graphicx}
\usepackage{courier}
\usepackage{booktabs}
\usepackage{tabularx}
\usepackage{microtype}
% \usepackage[T1]{fontenc}
% \renewcommand\tabularxcolumn[1]{m{#1}}% vertically center
\newcolumntype{Y}{>{\raggedright\arraybackslash}X} % A regular tabularx column but left justified instead of full justified.
% -------------------------------------------------------
% FROM https://tex.stackexchange.com/questions/348651/c-code-to-add-in-the-document
\usepackage{listings}
\usepackage{xcolor}
\definecolor{mGreen}{rgb}{0,0.6,0}
\definecolor{mGray}{rgb}{0.5,0.5,0.5}
\definecolor{mPurple}{rgb}{0.58,0,0.82}
\definecolor{backgroundColour}{rgb}{0.95,0.95,0.92}

\lstdefinestyle{CStyle}{
  commentstyle=\color{mGreen},
  keywordstyle=\color{magenta},
  numberstyle=\tiny\color{mGray},
  stringstyle=\color{mPurple},
  basicstyle=\footnotesize\ttfamily,
  breakatwhitespace=false,         
  breaklines=true,                 
  captionpos=b,       
  keepspaces=true,
  numbersep=5pt,
  numbers=left,
  showspaces=false,                
  showstringspaces=false,
  showtabs=false,                  
  tabsize=2,
  language=C
}

\lstset{style=CStyle}
% -------------------------------------------------------
% \usepackage[left=2cm,right=2cm,top=2cm,bottom=2cm]{geometry}


% BUG REPORT TEMPLATE -----------------------------------
%% \noindent\begin{tabularx}{\textwidth}{rY}
%%   \toprule
%%   File & \\
%%   Line & \\
%%   Description & \\
%%   Number of Configurations & \\
%%   \midrule
%%   \multicolumn{2}{c}{Code Sample} \\
%% \end{tabularx}
%% \noindent\lstinputlisting[linerange=, firstnumber=]{code/}
%% \noindent\begin{tabularx}{\textwidth}{rY}
%%   \midrule
%%   Status & \\
%%   Remarks & \\
%%   \bottomrule
%% \end{tabularx}
% -------------------------------------------------------

% BUG REPORT TEMPLATE -----------------------------------
%% \noindent\begin{tabularx}{\textwidth}{rY}
%%   \toprule
%%   File & \\
%%   Line & \\
%%   Description & \\
%%   Number of Configurations & \\
%%   \midrule
%%   \multicolumn{2}{c}{Code Sample} \\
%% \end{tabularx}
%% \noindent\lstinputlisting[linerange=, firstnumber=]{code/}
%% \noindent\begin{tabularx}{\textwidth}{rY}
%%   \midrule
%%   Status & \\
%%   Remarks & \\
%%   \bottomrule
%% \end{tabularx}
% -------------------------------------------------------

\title{Toybox Bug Analysis -- infer}
\author{Austin Mordahl}
\begin{document}
\maketitle

\section{Introduction}
\noindent These bugs were generated by Infer v0.15.0 and Toybox 0.7.5. Bug reports are classified into the following categories:

\begin{center}
  \noindent\begin{tabularx}{0.75\textwidth}{rX}
    \toprule
    True & A bug which exists and 1) its existence is unintended, or 2) whether or not its existence is purposeful is undetermined. \\
    \midrule
    Technically True & A bug for which the content of the cppcheck bug report is true, but whose existence is intended. The difference between a False and Technically True bug report is that the former could theoretically be detected by a more sophisticated implementation of cppcheck. \\
    \midrule
    False & A bug cppcheck finds which, upon further inspection, does not exist in the code. For example, cppcheck indicating a variable is passed to a function without being initialized, when the variable is actually an out parameter and intialized within the function.\\
    \bottomrule
  \end{tabularx}
\end{center}


\pagebreak
\section{True Reports}

\section{Technically True Reports}

\section{False Reports}

\noindent\begin{tabularx}{\textwidth}{rY}
  \toprule
  File & grep.c\\
  Line & 184\\
  Description & The value read from \texttt{matches.rm\_so} was never initialized.\\
  Number of Configurations & 507\\
  \midrule
  \multicolumn{2}{c}{Code Sample} \\
\end{tabularx}
\noindent\lstinputlisting[linerange=178-190, firstnumber=178]{code/grep.c}
\noindent\begin{tabularx}{\textwidth}{rY}
  \midrule
  Status & False\\
  Remarks & Were \texttt{matches.rm\_so} a singular variable, infer would be correct, becuase the initialization of \texttt{matches.rm\_so} would be out of scope. However, \texttt{matches} is a struct which is in scope. Additionally, the else if clause checks whether \texttt{matches.rm\_so} exists; line 184 will not be reached if \texttt{matches.rm\_so} is not initialized.\\
  \bottomrule
\end{tabularx}

\pagebreak

\noindent\begin{tabularx}{\textwidth}{rY}
  \toprule
  File & xwrap.c\\
  Line & 389\\
  Description & resource acquired by call to \texttt{xopen\_stdio()} at line 389, column 19 is not released after line 389, column 3. \\
  Number of Configurations & 986\\
  \midrule
  \multicolumn{2}{c}{Code Sample} \\
\end{tabularx}
\noindent\lstinputlisting[linerange=330-342, firstnumber=330]{code/xwrap.c}
\noindent\begin{tabularx}{\textwidth}{rY}
  \midrule
  Status & \\
  Remarks & \texttt{xopen\_stdio()} automatically closes a file unless the \texttt{O\_CLOEXEC} flag is passed to it (behaves opposite other functions which open files). There are two calls to \texttt{xopen\_stdio()} which do not pass \texttt{O\_CLOEXEC}. The first is in \texttt{oneit.c}, line 99. Here, the file descriptors are kept open on purpose, redirecting \texttt{stdin}, \texttt{stdout}, and \texttt{stderr}. The same pattern is used in the second occurrence. in \texttt{getty.c}.\\
  \bottomrule
\end{tabularx}

\end{document}

